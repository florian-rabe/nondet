\documentclass{beamer}
%\usepackage{graphicx,url}
\usepackage{amsmath}
\usepackage{amssymb}
%\usepackage{xkeyval}
%\usepackage{multirow}
%\usepackage{mytikz}
%\usepgflibrary{shapes}
\usepackage{stmaryrd}
\usepackage{basics}
\usepackage{basics-slides}

\def\choice#1{\fold{\sqcap}{#1}}
\def\prop{\cn{bool}}
\def\pure#1{\mathit{pure}(#1)}
\def\refine#1#2{#1\stackrel{\mathrm{ref}}{\leftarrow}#2}
\def\refine#1#2{#1\stackrel{\mathrm{ref}}{\leftarrow}#2}

\def\der{\,\hlA{\vdash}\,}
\def\syneq{\,\hlA{\doteq}\,}

\def\sem#1{\llbracket#1\rrbracket}
\def\seme#1#2{\llbracket#1\rrbracket_{#2}}
\def\semr#1{\seme{#1}{\rho}}
\def\rc#1{#1^\leftarrow}
\def\pwr{\mathbb{P}}
\def\PP#1{\pwr^{\neq\es}#1}

\makecn{optimum}
\makecn{Input}
\makecn{Cand}
\def\inp{input}
\makecn{cost}
\makecn{genCand}
\makecn{minCost}
\makecn{extCand}
\makecn{type}
\makecn{List}
\makecn{foldr}
\makecn{map}
\makecn{Option}


\begin{document}

\title{How to calculate with nondeterministic functions}
\author{Richard Bird and \underline{Florian Rabe}}
\institute{Computer Science, Oxford University resp. University Erlangen-N\"urnberg}
\date{July 2019}
\begin{frame}
    \titlepage
\end{frame}

\section{Background}

\begin{frame}\frametitle{Calculate Functional Programs}
\begin{itemize}
\item Bird–Meertens formalism (Squiggol)
 \begin{itemize}
 \item derive functional programs from specifications
 \item use equational reasoning to calculate correct programs
 \item optimize along the way
 \end{itemize}
 Example: \[h\,(\fold\,f\,e\,xs) = \fold\,F\,(h\,e)\, xs\]
 try to solve for $F$ to get more efficient algorithm
\item Richard's textbooks on functional programming
 \begin{itemize}
 \item Introduction to Functional Programming, 1988
 \item Introduction to Functional Programming using Haskell, 1998
 \item Thinking Functionally with Haskell, 2014
 \end{itemize} 
\end{itemize}
\end{frame}

\begin{frame}\frametitle{History}
\begin{blockitems}{My background}
\item not algorithms or functional programming
\item formal systems (logics, type theories, foundations, DSLs, etc.)
\item design, analysis, implementation
\item applications to all STEM disciplines
\end{blockitems}

\begin{blockitems}{This work}
\item Richard encountered problem with an elementary example
\item He built bottom-up solution using non-deterministic functions
\item I got involved in working out the formal details of the calculus
\end{blockitems}
\lec{i.e., my contribution is arguably the less interesting part of this work :)}
\end{frame}

%%%%%%%%%%%%%%%%%%%%%%%%%%%%%%%%%%%%%%%%
\section{Overview}

\begin{frame}\frametitle{Summary}
\begin{blockitems}{Our Approach}
\item Specifications tend to have non-deterministic flavor
  \lec{even when specifying deterministic functions}
\item Program calculation with deterministic $\lambda$-calculus can be limiting
\item Our idea: 
 \begin{itemize}
 \item extend to $\lambda$-calculus with non-deterministic functions
 \item in a way that preserves existing notations and theorems
  \lec{works well}
 \item mostly following the papers by Morris and Bunkenburg
 \end{itemize}
\end{blockitems}

\begin{blockitems}{Warning}
 \item We calculate and execute only deterministic functions.
 \item We use non-deterministic functions only for specifications and intermediate values.
   \glec{calculus allows more but not explored here}
\end{blockitems}
\end{frame}

\begin{frame}\frametitle{Non-Determinism}
Kinds of function
\begin{itemize}
\item Function $A\to B$ is relation on $A$ and $B$ that is
 \begin{itemize}
 \item total (at least one output per input)
 \item deterministic (at most one output per input)
 \end{itemize}
\item Partial functions = drop totality
 \begin{itemize}
 \item very common in math and elementary CS
 \item can be modeled as option-valued total functions \[A\to \Option\,B\]
 \end{itemize}
\item Non-deterministic functions = drop determinism
 \begin{itemize}
 \item somewhat dual to partial functions, but much less commonly used
 \item can be modeled as nonempty-set-valued deterministic functions \[A\to \PP\, B\]
 \end{itemize}
\end{itemize}
\end{frame}

%%%%%%%%%%%%%%%%%%%%%%%%%%%%%%%%%%%%%%%%
\section{Motivation}

\begin{frame}\frametitle{A Common Optimization Problem}
Two-step optimization process
\begin{enumerate}
\item generate list of candidate solutions (from some input)
 \[\genCand: \Input\to \List\, \Cand\]
\item choose cheapest candidate from that list
\end{enumerate}
\[\minCost: \List\, \Cand \to \Cand\]
\[\optimum\,\inp = \minCost\,(\genCand\,\inp)\]

\begin{blockitems}{$\minCost$ is where non-determinism will come in}
\item $\minCost\,cs$ = \alert{some} $c$ with minimal cost among $cs$
\lec{non-deterministic}
\item for now: $\minCost\, cs$ = \alert{first} such $c$
\lec{deterministic}
\end{blockitems}
\end{frame}


\begin{frame}\frametitle{A More Specific Setting}
\[\genCand: \Input\to \List\, \Cand\]
\[\minCost: \List\, \Cand \to \Cand\]
\hrule
\vspace{.5cm}

\begin{itemize}
\item $\inp$ is some recursive data structure
\item candidates for bigger input are built from \\ candidates for smaller input
\item our case: $\inp$ is a list, and $\genCand$ is a fold over \inp
 \[\extCand\,x: \Cand \to \List\,\Cand\]
  \glec{extends candidate for $xs$ to candidate list for $x::xs$}
 \[\genCand\, (x::xs) = \extCand\, x\,(\genCand\,xs)\]
\end{itemize}
\end{frame}

\begin{frame}\frametitle{Idea to Derive Efficient Algorithm}
\begin{itemize}
\item Fuse $\minCost$ and $\genCand$ into a single fold
\item Greedy algorithm
\begin{itemize}
\item don't build all candidates, apply $\minCost$ once at the end
\item apply $\minCost$ early on, extend only optimal candidates
\end{itemize}
\item Not necessarily sound:\\
   \tb non-optimal candidates for small input \\
   \tb might extend to \\
   \tb optimal candidates for large input
\end{itemize}

\hrule
\color{gray}
\[\optimum\,\inp = \minCost\,(\genCand\,\inp)\]
\[\genCand\, (x::xs) = \extCand\, x\,(\genCand\,xs)\]
\[\genCand: \Input\to \List\, \Cand\]
\[\minCost: \List\, \Cand \to \Cand\]
\[\extCand\,x: Cand \to \List\,\Cand\]
\end{frame}

\begin{frame}\frametitle{Solution through Program Calculation}
Obtain a greedy algorithm from the specification
\begin{enumerate}
\item Assume \[\optimum\,\inp = \fold\, F\,c_0\,\inp\]
 \glec{($c_0$ is base solution for empty input)}
and try to solve for folding function $F$
\item<2-> Routine equational reasoning yields
\begin{itemize}
  \item solution:
    \[F\,x\,c = \minCost\,(\extCand\,x\,c)\]
  \item soundness condition:
    \[\optimum\,(x::xs) = F\,x\,(\optimum\,xs)\]
Intuition: solution $F\,x\,c$ for input $x::xs$ is \\ cheapest extension of solution $c$ for input $xs$
\end{itemize}
%  \item \ldots if optimal candidates for small input extend to optimal candidates for 
%   \[\minCost\,(\flatMap\,(\extCand\,x)\,cs) = \minCost\,(\extCand\,x\,(\minCost\,cs))\]
\end{enumerate}
\end{frame}

\begin{frame}\frametitle{A Subtle Problem}
{{\color{gray}
Soundness condition (from previous slide):
  \[F\,x\,c = \minCost\,(\extCand\,x\,c)\]
  \[\optimum\,(x::xs) = F\,x\,(\optimum\,xs)\]
optimal candidate for $x::xs$ must be \\ optimal extension of optimal candidate for $xs$
}}
\hrule
\vspace{.5cm}

Soundness condition is intuitive and common\\
but subtly stronger than needed:
\begin{itemize}
\item $\optimum$ and $F$ defined in terms of $\minCost$
\item Actually states:\\
  \alert{first} optimal candidate for $x::xs$ is \\ \alert{first} optimal extension of \alert{first} optimal candidate for $xs$
 \lec{rarely holds in practice}
\end{itemize}
\end{frame}

\begin{frame}\frametitle{What went wrong?}
What happens:
\begin{itemize}
\item Specification of $\minCost$ naturally non-deterministic
\item Using standard $\lambda$-calculus forces\\ artificial once-and-for-all choice to make $\minCost$ deterministic
\item Program calculation uses \underline{only equality}
 \lec{artificial choices must be preserved}
\end{itemize}

What should happen:
\begin{itemize}
\item Use $\lambda$-calculus with non-deterministic functions
\item $\minCost$ returns \alert{some} candidate with minimal cost
\item Program calculation uses \underline{equality and refinement}
\lec{gradual transition towards deterministic solution}
\end{itemize}

%Alternative formal framework
%\begin{itemize}
%\item Switch to relational framework
%\item Possible but major departure from standard $\lambda$-calculus
%\lec{especially bad in a funcional programming textbook}
%\end{itemize}
\end{frame}

%\begin{tabular}{l@{\;:\,}l}
%$\Input$   & $\type$ \\
%$\inp$     & $\List\Input$ \\
%$\Cand$    & $\type$ \\
%$\genCand$ & $\List\Input \to \List\Cand$ \\
%$\minCost$ & $\List\Cand \to \Cand$\\
%\end{tabular}

%%%%%%%%%%%%%%%%%%%%%%%%%%%%%%%%%%%%%%%%
\section{Formal System: Syntax}

\begin{frame}{Key Intuitions (Don't skip this slide)}
Changes to standard $\lambda$-calculus
\begin{itemize}
\item $A\to B$ is type of \textbf{non-deterministic} functions
\item Every term represents a \textbf{nonempty set} of possible values
\item<2-> \textbf{Pure} terms \underline{roughly} represent a single value
\item<3-> \textbf{Refinement} relation between terms of the same type: \\
 $\refine{s}{t}$ \tb iff \tb $s$-values are also $t$-values
\item<4-> Refinement is an order at every type, in particular
 \[\refine{s}{t} \tb\wedge\tb \refine{t}{s} \tb\impl\tb s\syneq t\]
 \glec{$\syneq$ is the usual equality between terms}
\item<5-> Refinement for functions
\begin{itemize}
\item point-wise: $\refine{f}{g}$ \tb iff\tb $\refine{f(x)}{g(x)}$ for all pure $x$
\item deterministic functions are minimal wrt refinement
\end{itemize}
\end{itemize}
\end{frame}

\begin{frame}\frametitle{Syntax: Type Theory}
\begin{commgrammar}
\gprod{A,B}{a}{base types (integers, lists, etc.)} \\
\galtprod{A\to B}{non-det. functions} \\
\gprod{s,t}{c}{base constants (addition, folding, etc.)} \\
\galtprod{x}{variables} \\
\galtprod{\lambda x:A.t}{function formation} \\
\galtprod{s\,t}{function application} \\
\galtprod{\choice{s,t}}{non-deterministic choice}
\end{commgrammar}

Typing rules as usual plus
\[\rul{\der s:A \tb \der t:A}{\der \choice{s,t}:A}\]
\end{frame}

\begin{frame}\frametitle{Syntax: Logic}
Additional base types/constants:
\begin{itemize}
 \item $\prop:\type$
 \item logical connectives and quantifiers as usual, e.g.,
   \[\rul{\der s:A \tb \der t:A}{\der s\syneq t:\prop}\]
 \item refinement predicate
   \[\rul{\der s:A \tb \der t:A}{\der \refine{s}{t}:\prop}\]
 \item purity predicate
   \[\rul{\der t:A}{\der \pure{t}:\prop}\]
\end{itemize}
\end{frame}

%%%%%%%%%%%%%%%%%%%%%%%%%%%%%%%%%%%%%%%%
\section{Formal System: Semantics}

\begin{frame}\frametitle{Semantics: Overview}
\begin{center}
\begin{tabular}{ll}
\hline
Syntax & Semantics \\
\hline
type $A$              & set $\sem{A}$ \\
context declaring $x:A$ & environment mapping $\rho: x\mapsto \sem{A}$\\
term $t:A$      & nonempty subset $\semr{t}\in\PP\sem{A}$\\
refinement $\refine{s}{t}$ & subset $\semr{s}\sq\semr{t}$\\
purity $\pure{t}$ for $t:A$  & $\semr{t}$ is generated by a single $v\in\sem{A}$\\
choice $\choice{s,t}$ & union $\semr{s}\cup\semr{t}$\\
\hline
\end{tabular}
\end{center}
\end{frame}

\begin{frame}\frametitle{Semantics: Functions}
Functions are interpreted as set-valued semantic functions:
\[\sem{A\to B} = \sem{A}\Rightarrow\PP\sem{B}\]
\glec{using $\Rightarrow$ for the usual set-theoretical function space}

Function application is monotonous wrt refinement:
\[\semr{f\,t} = \bigcup_{\phi\in\semr{f},\tau\in\semr{t}} \phi(\tau)\]

\onslide<2>{
The interpretation of a $\lambda$-abstractions is closed under refinements:
\[\semr{\lambda x:A.t} = \big\{\phi\,|\, \text{ for all } \xi\in\sem{A}:\;\phi(\xi)\sq \seme{t}{\rho,x\mapsto \xi}\big\}\]
%\in\sem{A}\Rightarrow\PP\sem{B}
\glec{contains all deterministic functions that return refinements of $t$}
}
\end{frame}

\begin{frame}\frametitle{Semantics: Purity and Base Cases}
For every type $A$, also define embedding $\sem{A}\ni \xi\mapsto \rc{\xi}\sq\sem{A}$
\begin{itemize}
 \item for base types: $\rc{\xi}=\{\xi\}$
 \item for function types: closure under refinement
\end{itemize}

Pure terms are interpreted as embeddings of singletons:
\[\semr{\pure{t}}=1 \tb\text{iff}\tb \semr{t}=\rc{\tau}\text{ for some } \tau\]

\begin{itemize}
\item Variables
\[\seme{x}{\rho} = \rc{\rho(x)}\]
\glec{note: $\rho(x)\in\sem{A}$, not $\rho(x)\sq\sem{A}$}
\item Base types: as usual
\item Base constants $c$ with usual semantics $C$:
  \[\semr{c}=\rc{C}\]
  \glec{straightforward if $c$ is first-order}
\end{itemize}
\end{frame}

%%%%%%%%%%%%%%%%%%%%%%%%%%%%%%%%%%%%%%%%
\section{Formal System: Proof Theory}

\begin{frame}\frametitle{Overview}
\begin{blockitems}{Akin to standard calculi for higher-order logic}
\item Judgment $\Gamma\der F$ for a context $\Gamma$ and $F:\prop$
\item Usual axioms/rules for equality and propositional connectives
\lec{modifications needed when variable binding is involved}
\item Intuitive axioms/rules for choice and refinement
 \lec{technical difficulty to get purity right}
\end{blockitems}

\begin{blockitems}{Multiple equivalent axiom systems}
\item In the sequel, no distinction between primitive and derivable rules
\item Very subtle in practice to prove derivability of rules
 \lec{formalization in logical framework helps}
\end{blockitems}
\end{frame}

\begin{frame}\frametitle{Refinement and Choice}
\begin{itemize}
\item General properties of refinement
\begin{itemize}
\item $\refine{s}{t}$ is an order (wrt $\syneq$)
\item characteristic property:
  \[\refine{s}{t} \tb\text{ iff }\tb \refine{u}{s} \text{ implies } \refine{t}{u} \text{ for all } u\]
\end{itemize}
\item<2-> General properties of choice
\begin{itemize}
\item $\choice{s,t}$ is associative, commutative, idempotent (wrt $\syneq$)
\item no neutral element
 \glec{we do not have an undefined term with $\semr{\bot}=\es$}
\end{itemize}
\item<3-> Refinement of choice
\begin{itemize}
\item $\refine{u}{\choice{s,t}}$ refines to a pure term $u$ iff $s$ or $t$ does
\item in particular, $\refine{t_i}{(\choice{t_1,t_2})}$
\end{itemize}
\end{itemize}
\end{frame}

\begin{frame}\frametitle{Rules for Purity}
\begin{itemize}
\item Purity predicate only present for technical reasons
\item Pure are
\begin{itemize}
\item base constants applied to any number of pure arguments
\item $\lambda$-abstractions
\end{itemize}
 \glec{and thus all terms without $\sqcap$}
\item Syntactic vs. semantic approach
\begin{itemize}
 \item Semantic = use rule
  \[\rul{\der\pure{s} \tb \der s\syneq t}{\der\pure{t}}\]
   \lec{thus $\choice{1,1}$ is pure}
 \item literature uses syntactic rules like ``variables are pure''
   \lec{easier at first, trickier in the details}
\end{itemize}
\end{itemize}
\end{frame}


\begin{frame}\frametitle{Rules for Function Application}
\begin{itemize}
\item Distribution over choice:
\[\der f\,(\choice{s,t}) \syneq \choice{(f\,s),(f\,t)}\]
\[\der (\choice{f,g})\,t \syneq \choice{(f\,t),(g\,t)}\]
\item Monotonicity wrt refinement:
\[\rul{\der \refine{f'}{f} \tb \refine{t'}{t}}{\der\refine{f'\,t'}{f\,t}}\]
%\glec{accordingly for function composition}
\item Characteristic property wrt refinement:
 \[\refine{u}{f\,t} \tb\text{ iff }\tb \refine{f'}{f},\;\refine{t'}{t},\; \refine{u}{f'\,t'}\]
\end{itemize}

%\begin{itemize}
%\item intuition: bound variables are meant to range over values
%\item insert purity assumptions in the right places
%% \glec{alternatively, syntactic approach uses a single non-semantic rule for that}
%\end{itemize}
\end{frame}


\begin{frame}\frametitle{Beta-Conversion}
Intuition: bound variable is pure, so only substitute with pure terms
\[\rul{\der t:A \tb \pure{t}}{\der (\lambda x:A.t)\,s \syneq t[x/s]}\]

Counter-example if we omitted the purity condition
\begin{itemize}
\item Wrong:
\[(\lambda x:\Z.x+x)(\choice{1,2})\syneq (\choice{1,2})+(\choice{1,2})\syneq\choice{2,3,4}\]
\item Correct:
\[(\lambda x:\Z.x+x)\,(\choice{1,2})\syneq \choice{((\lambda x:\Z.x+x)\,1),((\lambda x:\Z.x+x)\,2)}\syneq\choice{2,4}\]
\end{itemize}
\lec{Computational intuition: no lazy resolution of non-determinism}
\end{frame}

\begin{frame}\frametitle{Xi-Conversion}
\begin{itemize}
\item Equality conversion under a $\lambda$, congruence rule for binders
\item Usual formulation
\[\rul{x:A\der f\syneq g}{\der \lambda x:A.f\syneq \lambda x:A.g}\]
\item Adjusted: bound variable is pure, so add purity assumption when traversing into a binder
\[\rul{x:A,\;\pure{x}\der f\syneq g}{\der \lambda x:A.f\syneq \lambda x:A.g}\]
 \glec{needed to discharge purity conditions of the other rules}
\end{itemize}
\lec{Computational intuition: functions can assume arguments to be pure}
\end{frame}

\begin{frame}\frametitle{Eta-Conversion}
Because $\lambda$-abstractions are pure, $\eta$ can only hold for pure functions
\[\rul{\der f:A\to B \tb \der\pure{f}}{\der f \syneq \lambda x:A.(f\,x)}\]

Counter-example if we omitted the purity condition:
\begin{itemize}
\item Wrong:
\[\choice{f,g} \syneq \lambda x:\Z.(\choice{f,g})\,x\syneq \lambda x:\Z.\choice{(f\,x),(g\,x)}\]
\item Correct: \[\refine{\choice{f,g}}{\lambda x:\Z.\choice{(f\,x),(g\,x)}}\]
but not the other way around
\end{itemize}
\lec{Computational intuition: choices under a $\lambda$ are resolved fresh each call}

%\hrule
%
%Accordingly for functional extensionality:
%\[\rul{\der f:A\to B \tb \der g:A\to B\tb\der\pure{f}\tb\der\pure{g}\tb x:A,\;\pure{x}\der f\,x \syneq g\,x}{\der f\syneq g}\]
\end{frame}

%%%%%%%%%%%%%%%%%%%%%%%%%%%%%%%%%%%%%%%%
\section{Formal System: Meta-Theorems}

\begin{frame}\frametitle{Overview}
Soundness
\begin{itemize}
\item If $\der F$, then $\semr{F}=1$
\item In particular: if $\der\refine{s}{t}$, then $\semr{s}\sq\semr{t}$.
\end{itemize}

Consistency
\begin{itemize}
\item $\der F$ does not hold for all $F$
\end{itemize}

Completeness
\begin{itemize}
\item Not investigated at this point
\item Presumably similar to usual higher-order logic
\end{itemize}
\end{frame}

%%%%%%%%%%%%%%%%%%%%%%%%%%%%%%%%%%%%%%%%
\section{Conclusion}

\begin{frame}\frametitle{Revisiting the Motivating Example}
\begin{itemize}
\item Applied to many examples in forthcoming textbook
  \glec{Algorithm Design using Haskell, Bird and Gibbons}
\item Two parts on greedy and thinning algorithms
\item Based on two non-deterministic functions
 \[\cn{MinWith}:\List\,A\to(A\to B)\to (B\to B\to\prop)\to A\]
 \[\cn{ThinBy}:\List\,A\to(A\to A\to\prop)\to \List\,A\]
\item $\minCost$ from motivating example defined using $\cn{MinWith}$
\item Soundness conditions for greedy algorithms can be proved for many practical examples
\end{itemize}
\end{frame}

%\begin{frame}\frametitle{Open Issue}
%\end{frame}

\begin{frame}\frametitle{Summary}
\begin{itemize}
\item Program calculation can get awkward if non-deterministic specifications are around
 \lec{e.g., minimal wrt to cost, or thinning wrt order}
\item Elegant solution by allowing for non-deterministic functions
\item Minimally invasive
 \begin{itemize}
 \item little new syntax
 \item old syntax/semantics embeddable
 \item only minor changes to rules 
 \item some subtleties but manageable
  \glec{formalization in logical framework helps}
 \end{itemize}
\item Many program calculation principles carry over
 \lec{deserves systematic attention}
\end{itemize}
\end{frame}
\end{document}

